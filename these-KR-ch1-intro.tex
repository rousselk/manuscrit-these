%%%%%%%%%%%%%%%%%%%%%%%%%%%%%%%%%%%%%%%%%%%%%%%%%%%%%%%%%%%%%%%%%%%%%%%%%%%%%%%%
%%%                               80 COLONNES                                %%%
%%%%%%%%%%%%%%%%%%%%%%%%%%%%%%%%%%%%%%%%%%%%%%%%%%%%%%%%%%%%%%%%%%%%%%%%%%%%%%%%

\chapter{Introduction}
\label{ChIntro}

\vspace{-2mm}

Les capteurs et actionneurs sans-fil (\lang{``Wireless Sensors and
Actuators''})~--- qui sont l'objet même des travaux de cette thèse~---
sont en fait des <<~nano-ordinateurs~>> embarqués, regroupant unité
centrale, interface réseau sans-fil (radio), et divers périphériques
leur permettant d'interagir avec leur environnement (capteurs et
actionneurs), sur une carte électronique dont la taille ne dépasse
en général pas celle d'une carte de crédit.

Une de leur spécificités majeures est de dépendre de batteries (piles
ou autres) pour leur alimentation. Cette batterie étant parfois difficile
voire impossible à changer, ces nano-ordinateurs doivent être conçus
et programmés pour \emph{minimiser au maximum leur consommation d'énergie}~:
leur durée de fonctionnement, et parfois même leur durée de vie, en dépend.

Ces nano-ordinateurs sont couramment appelés \nom{capteurs sans-fil}~---
les appareils équipés d'actionneurs étant nettement moins nombreux et
utilisés~---, \nom{n{\oe}uds} ou par le terme anglo-saxon
\emph{\lang{``mote''}}.

Le terme de <<~noeud~>> est ici particulièrement révélateur, car ces
appareils sont par nature destinés à fonctionner en réseau, via le
médium radio. Ce sont ces réseaux de noeuds qui sont appelés
\nom{Réseaux de Capteurs Sans-Fil} (\emph{``Wireless Sensor Networks''}
ou \nom{WSN}, selon le terme anglo-saxon).

L'architecture et le fonctionnement de ces capteurs sans-fil et de leurs
réseaux, et les défis qu'ils posent, seront définis et expliqués de façon
détaillée en section \vref{SubsecDefWSN} du prochain chapitre.

Ces réseaux de capteurs sans-fil, en s'interconnectant entre eux
et avec les réseaux globaux (WAN~: \lang{Wide Area Network}), ont permis
l'apparition de la notion plus récente d'\nom{Internet des Objets}
(\nom{IoT}~: \lang{Internet of Things}), <<~organisme~>> dont ils
constituent les cellules.

Il s'agit à l'heure actuelle d'un sujet de recherche et de développement
extrêmement vaste, actif et prometteur. Le développement rapide de cet IoT
permet l'apparition et la mise en place d'une multitude d'applications
nouvelles.

Ces applications, de plus en plus variées, riches et complexes, augmentent
encore l'intérêt de plates-formes logicielles (c'est-à-dire de
\nom{systèmes d'exploitation}, en anglais \lang{``Operating System''}~:
\nom{OS}) fiables, fonctionnelles, performantes et adaptées aux noeuds
de ces WSN constituant le fondement de l'IoT.

%%%%%%%%%%%%%%%%%%%%%%%%%%%%%%%%%%%%%%%%%%%%%%%%%%%%%%%%%%%%%%%%%%%%%%%%%%%%%

\bigskip

Les domaines d'application des réseaux de capteurs sans-fil, et par
extension de l'Internet des Objets, sont extrêmement étendus.
Deux livres \cite{LivreDargie2010} \cite{LivreAkyildiz2010}
détaillent différentes applications déjà existantes et exploitées.

On peut notamment citer~:

\begin{itemize}

\item des applications militaires,

\item des applications industrielles,

\item des applications environnementales,

\item des applications domotiques,

\item des applications à la santé.

\end{itemize}

Ces deux derniers domaines d'application des WSN sont ceux auxquels nous
nous intéressons spécifiquement dans la présente thèse.

Nous disposons notamment, au LORIA~/ INRIA Nancy Grand-Est,
d'un projet d'appartement intelligent pour l'assistance à la personne
\cite{AppartIntelligent}. Ce projet, constamment en cours de développement
et de perfectionnement, est utilisé de façon intensive par les différentes
équipes de recherche du site, pour le développement et le test
d'applications diverses, aussi bien académiques qu'industrielles,
principalement pour l'aide au maintien à domicile des personnes âgées
et~/ ou dépendantes. Ce projet relève à la fois de l'application
domotique et de l'application de santé, tout comme le projet LAR
que nous allons détailler dans le chapitre \ref{ChCtxProb}.

%%%%%%%%%%%%%%%%%%%%%%%%%%%%%%%%%%%%%%%%%%%%%%%%%%%%%%%%%%%%%%%%%%%%%%%%%%%%%

\bigskip

Comme la plupart des systèmes informatiques connectés, les capteurs sans-fil
ont recours à des piles protocolaires pour gérer l'envoi et la réception de
données. Dans leur cas, il s'agit d'émettre et de recevoir ces données
sous forme de trames transmises sur le médium radio~--- par définition peu
fiable, et souvent sujet à des perturbations. La plupart des WSN actuels
accèdent au médium radio selon le standard IEEE 802.15.4
\cite{IEEE802154-2011}.

Pour pallier les problèmes liés à l'instabilité de ce médium radio, de
nombreux protocoles MAC (\lang{``Media Access Control''}) ont été
développés. Le protocole 802.15.4 en propose lui-même deux versions,
mais leurs limitations ont poussé la communauté académique et industrielle
à développer de nombreux protocoles alternatifs, basés sur des principes
et des techniques différents.

\medskip

Pour faciliter la programmation et l'exploitation de ces systèmes spéciaux
que sont les capteurs sans-fil, il existe des systèmes d'exploitation
spécifiques, prenant en compte leurs spécificités~: capacités très limitées,
fonctionnement sur batterie faisant de la consommation énergétique un
enjeu majeur, communication par radio. Nombre de ces plates-formes
logicielles dédiées ont été conçues et sont exploitées à l'heure actuelle,
mais les plus utilisés au moment où nous écrivons ces lignes sont
Tiny OS \cite{TinyOS} et surtout Contiki OS \cite{ContikiOS}.
Ces plates-formes logicielles sont fournies avec leurs propres piles
réseau intégrées, qui sont donc la cheville ouvrière du fonctionnement
concret des communications sur les réseaux de capteurs sans-fil.

\newpage

Si de nombreux travaux de recherche ont, comme nous l'avons dit, été
menés pour développer des protocoles MAC de plus en plus performants
pour exploiter au mieux le médium radio et contourner ses limitations,
tout en préservant au maximum les ressources énergétiques des capteurs
sans-fil, les résultats de ces travaux n'ont malheureusement guère
été concrètement implantés et diffusés à l'heure actuelle dans les piles
réseaux des systèmes d'exploitation spécialisés~: celles-ci ne comportent
le plus souvent que le protocole MAC du standard IEEE 802.15.4, plus
éventuellement quelques protocoles simples et~/ ou anciens ne représentant
nullement l'état de l'art en la matière.

La seule exception notable à cette situation est la présence en standard
du protocole ContikiMAC \cite{ContikiMAC} dans la pile réseau des versions
récentes Contiki OS. Ce protocole, s'il est récent et performant, repose
toutefois sur des principes de fonctionnement somme toute classiques
(bien qu'optimisés) impliquant certaines limites. De nombreux travaux
d'implantation de protocoles différents~--- notamment sur leurs principes
de fonctionnement~--- restent encore à entreprendre.

Ainsi, à l'heure actuelle, les couches basses des piles réseau spécialisées
pour les capteurs sans-fil, notamment celles intégrées aux plates-formes
logicielles dédiées, constituent un <<~goulot d'étranglement~>> pour la
performance des communications entre \lang{motes}, les couches MAC (et
plus généralement les couches basses) semblant avoir été peu prioritaires
dans les efforts de développement et d'implantation des piles réseau
des OS pour WSN.

\medskip

\emph{Cette thèse a ainsi pour but de permettre d'obtenir des améliorations
significatives pour ces couches basses, tant sur le plan des performances
que de l'optimisation de la consommation énergétique, via un effort de
recherche et d'implantation des piles réseau dédiées, notamment en exploitant
au mieux les fonctionnalités offertes par les plates-formes logicielles
spécialisées dans les capteurs sans-fil.}

Si les OS les plus utilisés que sont Tiny OS \cite{TinyOS} et Contiki OS
\cite{ContikiOS} n'offrent pas les fonctionnalités nécessaires, d'autres
plates-formes logicielles moins répandues, mais plus performantes et récentes,
offrent notamment des mécanismes avancés de gestion des interruptions,
un modèle multitâche préemptif, et des fonctionnalités temps-réel (notamment
en exposant au mieux~--- via une API adaptée~--- les \lang{timers} matériels
présents dans les microntrôleurs équipant les capteurs sans-fil).
Ces fonctionnalités sont notamment très utiles pour améliorer la qualité
de service (QdS) des réseaux. Parmi les systèmes pour capteurs sans-fil
offrant de telles capacités, on pourra entre autres citer Nano-RK
\cite{NanoRK} ou RIOT OS \cite{RIOT}.

Un autre mécanisme lié à l'OS particulièrement intéressant pour l'économie
d'énergie est la présence d'un noyau fonctionnant en mode \lang{``tickless''},
c'est-à-dire permettant de ne faire fonctionner l'appareil que quand cela
est strictement nécessaire. 

\emph{Nous nous proposons d'exploiter toutes ces fonctionnalités avancées
offertes par ces OS dédiés pour tenter d'implanter l'état de l'art en matière
de protocoles MAC, et ainsi obtenir de meilleurs résultats en termes de
performances de communication et d'économies d'énergie.}

\medskip

Notons que pour des raisons juridiques aussi bien que techniques~---
possibilité de modifier et d'améliorer le c{\oe}ur et les différents
composants du système selon nos besoins~--- nous n'envisagerons dans
la présente thèse uniquement l'utilisation des systèmes d'exploitation~---
et plus généralement des logiciels~--- à licence libre et \lang{open
source}.

%%%%%%%%%%%%%%%%%%%%%%%%%%%%%%%%%%%%%%%%%%%%%%%%%%%%%%%%%%%%%%%%%%%%%%%%%%%%%

\subsection*{Objectifs}

Les travaux de la présente thèse ont \emph{les principaux objectifs}
suivants~:

\begin{enumerate}

\item Un \emph{état de l'art des différents protocoles MAC créés par la
recherche} académique ou industrielle, et surtout un passage en \emph{revue
des principaux systèmes d'exploitation utilisés dans le cadre des réseaux
de capteurs sans-fil}, en analysant leurs fonctionnalités, déterminant ainsi
\emph{quels sont les mieux adaptés au développement de couches basses}
(notamment MAC) avancées et performantes.

\item Le \emph{choix de la plate-forme logicielle la mieux adaptée} pour
le développement de ces couches basses, avec notamment une \emph{analyse
critique des piles réseau~--- notamment de l'API et des drivers radio---
de ces plates-formes spécialisées (Contiki et RIOT OS)}.

\item Une \emph{implantation du protocole S-CoSenS sur le système RIOT OS}~---
s'inscrivant dans un effort plus global destiné à \emph{fournir une couche
MAC performante à la pile réseau de cette plate-forme logicielle}~--- suivi
d'une \emph{comparaison avec l'implantation standard de ContikiMAC} sur
Contiki OS, notamment en présence d'un trafic réseau intense~; et enfin
la \emph{proposition d'idées d'améliorations algorithmiques à apporter
aux protocoles MAC}.

\item La \emph{validation sur plates-formes réelles} des résultats de nos
expérimentations~--- effectuées jusqu'alors par simulation~/ émulation~---
suite à la \emph{découverte d'un problème d'inexactitude temporelle dans
l'outil de simulation Cooja~/ MSPSim}, avec une analyse des problèmes
rencontrés, et la fourniture d'autant de détails techniques et des pistes
de résolution possibles pour aider à la résolution ultérieure des
difficultés rencontrées.

\end{enumerate}

%%%%%%%%%%%%%%%%%%%%%%%%%%%%%%%%%%%%%%%%%%%%%%%%%%%%%%%%%%%%%%%%%%%%%%%%%%%%%

\subsection*{Structure}

Le présent manuscrit de thèse est organisé de la façon suivante~:

\begin{itemize}

\item Après ce présent chapitre d'introduction, le chapitre
\vref{ChCtxProb} donne les définitions techniques nécessaires 
à la bonne compréhension du sujet, développe les différentes applications
possibles des WSN et de l'IoT, puis présente le contexte de la thèse,
et enfin la problématique que celle-ci se propose de résoudre, en commençant
à détailler nos pistes de travail.

\item Le chapitre \vref{ChEtatArt} présente l'état de l'art sur le
protocole IEEE 802.15.4 sur lequel reposent les réseaux de capteurs sans-fil
actuels~; une présentation des différents axes de recherche et des exemples
significatifs de protocoles MAC développés par la communauté pour suppléer
aux limitations du protocole MAC du standard 802.15.4~; et enfin, nous
faisons une première contribution sous la forme d'une revue
(\lang{``survey''}) des systèmes d'exploitation (OS) spécialisés dans
le domaine des WSN, en détaillant sucessivement leurs points forts et
leurs limitations, et par là-même leur adaptation au développement de
protocoles MAC avancés et performants.

\item Le chapitre \vref{ChPFLogicielles} détaille notre recherche
d'une plate-forme logicielle (OS) adaptée à nos travaux de recherche sur
les protocoles MAC à hautes performances, montre nos contributions au
développement de la plate-forme performante et novatrice (RIOT OS) que
nous avons choisie, et propose dans ce cadre une étude critique de sa
nouvelle pile réseau de RIOT OS (<<~gnrc~>>).

\item Le chapitre \vref{ChProtocolesMAC} montre les résultats de nos
premières expériences en comparant les implantations d'un protocole
hybride, S-CoSenS, mis au point au sein de notre équipe, à celle
du protocole ContikiMAC, référence largement utilisée dans la communauté.
Ces comparaisons montrent en particulier le comportement de ces deux
implantations de protocoles face à une montée en charge intensive
du trafic réseau. Nous proposons également plusieurs techniques
susceptibles d'améliorer la robustesse des protocoles MAC~/ RDC,
notamment en complétant l'interface avec la couche 1 (pilotes
des émetteurs~/ récepteurs radio) afin d'influer dynamiquement
sur des paramètres liés à l'écoute du médium.

\item Le chapitre \vref{ChValidation} présente tout d'abord les
inexactitudes d'ordre temporel que nous avons découvertes dans les
résultats fournis par Cooja, l'un des simulateurs de WSN les plus utilisés~;
nous y montrons nos contributions sous la forme d'une analyse des
limitations de ce dernier comme outil d'évaluation de performances, et des
conséquences possibles sur la validité et la justesse des travaux basés
sur ces simulations (y compris nos propres travaux)~; nous fournissons
enfin des pistes sérieuses quant aux causes du problème, et aux moyens
de le contourner ou d'y remédier.
Sont ensuite détaillés les travaux de validation prévus sur matériel
pour valider de façon indiscutable nos précédentes expériences, ainsi
que pour tester la montée en charge de S-CoSenS sur un réseau de forte
taille. Nous continuons en décrivant nos premières expériences sur
la plate-forme matérielle de test choisie, le \lang{testbed} IoT-LAB.
Nous détaillons enfin les problèmes techniques nous ayant empêché de
terminer de mener à bien ces travaux, en tentant de fournir le maximum
de pistes techniques pour faciliter leur résolution future.

\item Enfin, le chapitre \vref{ChConcluPerspec} termine ce
manuscrit de thèse en présentant nos conclusions générales, et
en discutant des perspectives pouvant faire suite à nos travaux.

\end{itemize}


%%%%%%%%%%%%%%%%%%%%%%%%%%%%%%%%%%%%%%%%%%%%%%%%%%%%%%%%%%%%%%%%%%%%%%%%%%%%%
%%%                    FIN DU CHAPITRE "INTRODUCTION"                     %%%
%%%%%%%%%%%%%%%%%%%%%%%%%%%%%%%%%%%%%%%%%%%%%%%%%%%%%%%%%%%%%%%%%%%%%%%%%%%%%


