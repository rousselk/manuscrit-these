%%%%%%%%%%%%%%%%%%%%%%%%%%%%%%%%%%%%%%%%%%%%%%%%%%%%%%%%%%%%%%%%%%%%%%%%%%%%%%%%
%%%                               80 COLONNES                                %%%
%%%%%%%%%%%%%%%%%%%%%%%%%%%%%%%%%%%%%%%%%%%%%%%%%%%%%%%%%%%%%%%%%%%%%%%%%%%%%%%%

\chapter*{Glossaire}
\label{AnxGlossaire}

%%%%%%%%%%%%%%%%%%%%%%%%%%%%%%%%%%%%%%%%%%%%%%%%%%%%%%%%%%%%%%%%%%%%%%%%%%%%%

\section*{0--9}

\begin{description}

\item[6LoWPAN] \lang{``IPv6 over Low-Power Wireless
                       Personal Area Network''} \\
Acronyme anglo-saxon, désignant une version <<~allégée~>> du protocole IP
version 6, adaptée aux réseaux de capteurs sans-fil, notamment aux
ressources limitées des matériels (n{\oe}uds) les composant.

\item[802.15.4] \ \\
Terme raccourci désignant le standard IEEE 802.15.4 (et en général ses
différentes annexes et extensions). Il définit une technologie sur laquelle
repose de nombreux réseaux de capteurs sans-fil, notamment ceux auxquels
les travaux de la présente thèse se sont intéressés.

\end{description}

%%%%%%%%%%%%%%%%%%%%%%%%%%%%%%%%%%%%%%%%%%%%%%%%%%%%%%%%%%%%%%%%%%%%%%%%%%%%%

\section*{A}

\begin{description}

\item[Actuator] \ \\
Terme anglo-saxon, désignant en français un \nom{actionneur} capable
d'influer sur un phénomène physique environnemental. La présence de tels
actionneurs sur un n{\oe}ud équipe d'un émetteur~/ récepteur radio est
ce qui le fait désigner sous le terme de <<~actionneur sans-fil~>>
(\lang{``Wireless Actuator''} en anglais). Les n{\oe}uds équipés
d'actionneurs sont sauf exception également équipés de capteurs, ce qui fait
que le terme de <<~capteur sans-fil~>> (\lang{``Wireless Sensor''}) reste
presque toujours utilisé, même en présence d'actionneurs.

\item[API] \lang{``Application Programming Interface''} \\
Acronyme anglo-saxon, se traduisant en français par \nom{<<~Interface de
Programmation d'Applications~>>}. Désigne l'ensemble des fonctions, types,
classes, etc., offerts aux développeurs pour construire leurs applications
au sein d'un environnement logiciel particulier (par exemple~: système
d'exploitation, ou bibliothèque offrant un jeu de fonctionnalités).

\end{description}

%%%%%%%%%%%%%%%%%%%%%%%%%%%%%%%%%%%%%%%%%%%%%%%%%%%%%%%%%%%%%%%%%%%%%%%%%%%%%

\section*{B}

\begin{description}

\item[BAN] \lang{``Body Area Network''} \\
Acronyme anglo-saxon, se traduisant en français par \nom{<<~Réseau
d'\'Etendue Corporelle~>>}. Désigne un type particulier de PAN, conçu pour
regrouper des capteurs et actionneurs sur le corps d'un individu.
Ce terme possède~--- notamment dans le cadre de la présente thèse~--- une
forte connotation médicale et d'aide à la personne (malade, agée,
dépendante, etc.).

\item[Beacon] \ \\
Terme anglo-saxon, se traduisant en français par \nom{<<~balise~>>}.
Désigne une trame ne contenant pas de données applicatives
(\lang{``payload''}), mais servant à la synchronisation des différents
n{\oe}uds d'un réseau de capteurs sans-fil. De telles <<~balises~>> ou
\lang{``beacons''} sont notamment utilisées par les protocoles MAC~/ RDC
S-CoSenS et le protocole standard IEEE 802.15.4 (dans un de ses deux modes
de fonctionnement).

\end{description}

%%%%%%%%%%%%%%%%%%%%%%%%%%%%%%%%%%%%%%%%%%%%%%%%%%%%%%%%%%%%%%%%%%%%%%%%%%%%%

\section*{C}

\begin{description}

\item[CCA] \lang{``Clear Channel Assessment''} \\
Acronyme anglo-saxon, désignant la procédure consistant à vérifier la
disponibilité du médium physique, juste avant l'envoi d'un message.

\item[Contiki (OS)] \ \\
L'une des plates-formes logicielles spécialisées parmi les plus utilisées
dans le domaine des WSN, étant actuellement la référence de fait dans le
domaine. Elle a fait l'objet de plusieurs travaux et études
dans le présente thèse.

\item[ContikiMAC] \ \\
Protocole MAC~/ RDC, conçu par le créateur de Contiki OS, et désormais
utilisé par défaut avec ce dernier depuis plusieurs années. Il s'agit
d'un protocole asynchrone de type LPL, ayant fait l'objet de plusieurs
optimisations conceptuelles et implantatoires. Compte-tenu de l'influence
majeure de Contiki OS dans le domaine des WSN, ce protocole est devenu une
référence par défaut dans le domaine, face à laquelle il est courant de
tester les nouveaux protocoles MAC~/RDC.

\item[CoSenS] \ \\
Protocole MAC, conçu au sein de notre équipe de recherche (Madynes).
Est la base et le précurseur de S-CoSenS, lequel en est une extension
majeure ajoutant la gestion du RDC.

\item[CPU] \lang{``Central Processing Unit''} \\
Acronyme anglo-saxon, désignant un \nom{c{\oe}ur de microprocesseur},
autour duquel (ou desquels) est construit tout système informatique.

Ces c{\oe}urs CPU peuvent être associés avec divers périphériques au
sein d'un même circuit, comme un microcontrôleur.

\item[CSMA/CA] \lang{``Carrier Sense Multiple Access
                       with Collision Avoidance''} \\
Acronyme anglo-saxon, désignant une procédure, basée sur la contention,
destinée à gérer l'envoi correct d'un paquet de données sur le médium
radio.

À noter que cette procédure, utilisée par la couche MAC standard du
standard IEEE 802.15.4, est différente d'une procedure homonyme utilisée
dans le standard WiFi (IEEE 802.11).

\end{description}

%%%%%%%%%%%%%%%%%%%%%%%%%%%%%%%%%%%%%%%%%%%%%%%%%%%%%%%%%%%%%%%%%%%%%%%%%%%%%

\section*{D}

\begin{description}

\item[Driver] \ \\
Terme anglo-saxon, se traduisant en français par \nom{<<~pilote~>>}.
\'Element logiciel, en général de bas niveau, permettant d'accéder aux
fonctionnalités d'un matériel donné (comme un périphérique).

\end{description}

%%%%%%%%%%%%%%%%%%%%%%%%%%%%%%%%%%%%%%%%%%%%%%%%%%%%%%%%%%%%%%%%%%%%%%%%%%%%%

\section*{E}

\begin{description}

\item[ED] \lang{``Energy Detection''} \\
Acronyme anglo-saxon, se traduisant en français par \nom{<<~Détection
d'énergie~>>}. Mesure chiffrée de la puissance du signal sur le médium
radio, notamment par rapport au bruit de fond. Sert souvent à déterminer
la disponibilité ou non de ce médium, lors d'une procédure de CCA.

\end{description}

%%%%%%%%%%%%%%%%%%%%%%%%%%%%%%%%%%%%%%%%%%%%%%%%%%%%%%%%%%%%%%%%%%%%%%%%%%%%%

\section*{F}

\begin{description}

\item[FDMA] \lang{``Frequency Division Multiple Access''} \\
Acronyme anglo-saxon, désignant une procédure, basée sur le multiplexage
fréquentiel, destinée à gérer l'envoi correct d'un paquet de données sur
le médium radio.

Employée par de nombreux protocoles MAC alternatifs, conçus par la
communauté académique ou industrielle, mais aussi par l'amendement
802.15.4e du standard IEEE.

\item[FFD] \lang{``Full Function Device''} \\
Acronyme anglo-saxon, issu du standard IEEE 802.15.4, désignant un matériel
suffisamment puissant pour jouer le rôle de coordinateur de PAN~--- et par
extension, bien souvent de routeur dans les couches supérieures de la pile
protocolaire, voire même assurer d'autres fonctions de plus haut niveau
comme par exemple des opérations de calcul. Ce terme est opposé à celui
de RFD.

\item[FPGA] \lang{``Field Programmable Gate Array''} \\
Acronyme anglo-saxon, désignant un circuit intégré dont l'organisation
interne, et donc les fonctionnalités, peuvent être programmées et
reprogrammées arbitrairement selon les besoins. Cela en fait de formidables
outils de prototypage matériel, mais aussi, avec l'augmentation de leur
puissance et la baisse de leur prix, de futures alternatives potentielles
aux circuits <<~dédiés~>> (comme les MCU par exemple) pour construire des
systèmes électroniques et informatiques, comme les n{\oe}uds de réseaux
de capteurs sans-fil.

\end{description}

%%%%%%%%%%%%%%%%%%%%%%%%%%%%%%%%%%%%%%%%%%%%%%%%%%%%%%%%%%%%%%%%%%%%%%%%%%%%%

\section*{G}

\begin{description}

\item[GPIO] \lang{``General Purpose Input/Output''} \\
Acronyme anglo-saxon, désignant une entrée~/ sortie à vocation généraliste.
Dans notre contexte, il s'agit d'une broche de MCU jouant ce rôle d'entrée~/
sortie librement utilisable et programmable.

\end{description}

%%%%%%%%%%%%%%%%%%%%%%%%%%%%%%%%%%%%%%%%%%%%%%%%%%%%%%%%%%%%%%%%%%%%%%%%%%%%%

\section*{H}

\begin{description}

\item[HAL] \lang{``Hardware Abstraction Layer''} \\
Acronyme anglo-saxon, se traduisant en français par \nom{<<~Couche
d'Abstraction Matérielle~>>}. Désigne le regroupement des éléments logiciels
dépendants du matériel (en général les pilotes ou \lang{``drivers''}) dans
une couche clairement séparée, permettant au reste du logiciel~--- en
général un système d'exploitation, ou un exécutif comparable~--- d'être
conçu de façon générique et donc plus facilement portable.

\end{description}

%%%%%%%%%%%%%%%%%%%%%%%%%%%%%%%%%%%%%%%%%%%%%%%%%%%%%%%%%%%%%%%%%%%%%%%%%%%%%

\section*{I}

\begin{description}

\item[I\textsuperscript{2}C] \lang{``Inter-Integrated Circuit''} \\
Acronyme anglo-saxon, désignant une technologie de bus série, reliant des
circuits au sein d'un même appareil électronique (parfois même au sein
d'un même microcontrôleur).

\item[IoT] \lang{``Internet of Things''} \\
Acronyme anglo-saxon, se traduisant en français par \nom{<<~Internet
des Objets~>>}. Ce terme désigne la possibilité des réseaux de capteurs
sans-fil de s'interconnecter entre eux et avec Internet, par l'utilisation
du protocole IP (\lang{``Internet Protocol''}), notamment en version 6
(IPv6), comme couche 3 de leur pile protocolaire.

\end{description}

%%%%%%%%%%%%%%%%%%%%%%%%%%%%%%%%%%%%%%%%%%%%%%%%%%%%%%%%%%%%%%%%%%%%%%%%%%%%%

\section*{J}

\begin{description}

\item[JTAG] \lang{``Joint Test Action Group''} \\
Acronyme anglo-saxon, désignant la norme IEEE 1149.1. Désigne (par abus de
langage) dans le présent manuscrit, la possibilité d'effectuer du déboguage
au sein du matériel lui-même. Dans notre contexte, il s'agit de déboguer
les programmes téléchargés au sein des n{\oe}uds de réseaux de capteurs
sans-fil \emph{lors} de leur exécution, ce qui facilité grandement la
compréhension et la résolution de bogues et autres problèmes.

\end{description}

%%%%%%%%%%%%%%%%%%%%%%%%%%%%%%%%%%%%%%%%%%%%%%%%%%%%%%%%%%%%%%%%%%%%%%%%%%%%%

\section*{L}

\begin{description}

\item[LAR] \lang{``Living Assistant Robot''} \\
Acronyme désignant le projet au sein duquel s'est déroulé cette thèse.

\item[LPL] \lang{``Low-Power Listening''} \\
Acronyme anglo-saxon, désignant une famille de protocoles MAC~/ RDC
aynchrones, basés sur la contention, où l'initiative de la transmission
des données est laissée aux n{\oe}uds émetteurs.

\item[LPP] \lang{``Low-Power Probing''} \\
Acronyme anglo-saxon, désignant une famille de protocoles MAC~/ RDC
aynchrones, basés sur la contention, où l'initiative de la transmission
des données est laissée aux n{\oe}uds receveurs, via des mécanismes
de transmissions régulières de <<~balises~>> afin de mieux économiser
l'énergie et exploiter la bande passante du médium radio.

\item[LQI] \lang{``Link Quality Indicator''} \\
Acronyme anglo-saxon, se traduisant en français par \nom{<<~Qualité de
Liaison~>>}. Mesure chiffrée de la qualité de transmission sur un médium,
dans le contexte qui nous intéresse le médium radio. Il s'agit d'un
indicateur de qualité de service.

\end{description}

%%%%%%%%%%%%%%%%%%%%%%%%%%%%%%%%%%%%%%%%%%%%%%%%%%%%%%%%%%%%%%%%%%%%%%%%%%%%%

\section*{M}

\begin{description}

\item[MAC] \lang{``Medium Access Control''} \\
Acronyme anglo-saxon, se traduisant en français par \nom{<<~Contrôle
d'Accès au Médium~>>}. Représente la couche 2 de la pile protocolaire
dans le modèle réseau OSI (ou précisément, la partie la plus basse de
cette couche 2).

Cette couche est chargée de contrôler l'accès au médium du réseau, dans
notre cas le médium radio, afin de s'assurer de la bonne qualité des
transmissions, et notamment de l'absence de collisions. \`A cette fin,
de nombreux protocoles ont été développés pour implanter cette couche 2~:
les protocoles MAC. Certains sont issus de standards comme IEEE 802.15.4,
d'autres sont le résultat de recherches académiques et/ou industrielles.

\item[MCU] \lang{``MicroController Unit''} \\
Acronyme anglo-saxon, désignant un \nom{Microcontrôleur}, c'est-à-dire
un circuit regroupant processeur central et plusieurs périphériques de
base intégrés, comme des \lang{timers}, des entrées-sorties avec
gestion de bus de communication intégrée, des convertisseurs entre
signaux analogiques et digitaux, etc.

De tels circuits sont utilisés massivement dans le domaine de l'informatique
embarquée, ainsi que dans les noeuds de WSN, dont ils constituent le
<<~c{\oe}ur~>>.

\item[Mote] \ \\
Terme anglo-saxon désignant~--- dans le cadre du présent manuscrit de
thèse~--- un n{\oe}ud d'un réseau de capteurs sans-fil.

\item[MPU] \lang{``Memory Protection Unit''} \\
Acronyme anglo-saxon, désignant une \nom{unité de protection mémoire}.
Il s'agit d'un circuit dédié, optionnellement intégré à un microcontrôleur,
afin de réserver certaines parties de la mémoire à certaines parties de
code, notamment les parties de code s'exécutant dans un <<~mode privilégié~>>
(parfois aussi appelé <<~superviseur~>>, <<~système~>>, etc.) que le
microcontrôleur est susceptible de fournir.

La présence d'une MPU dans un microcontrôleur facilite ainsi la détection
des erreurs liées à la mémoire (débordements, accès anormaux, etc.), et
aide grandement à la réalisation de systèmes fiables et robustes.

\end{description}

%%%%%%%%%%%%%%%%%%%%%%%%%%%%%%%%%%%%%%%%%%%%%%%%%%%%%%%%%%%%%%%%%%%%%%%%%%%%%

\section*{O}

\begin{description}

\item[OS] \lang{``Operating System''} \\
Acronyme anglo-saxon, se traduisant en français par \nom{<<~Système
d'Exploitation~>>}.

\end{description}

%%%%%%%%%%%%%%%%%%%%%%%%%%%%%%%%%%%%%%%%%%%%%%%%%%%%%%%%%%%%%%%%%%%%%%%%%%%%%

\section*{P}

\begin{description}

\item[PAN] \lang{``Personal Area Network''} \\
Acronyme anglo-saxon, se traduisant en français par \nom{<<~Réseau
d'\'Etendue Personnel~>>}. Désigne en général un réseau de capteurs
sans-fil élémentaire, regroupé autour de et contrôlé par un n{\oe}ud
coordinateur spécifique.

\item[PHY] <<~Couche PHYsique~>> \\
Diminutif désignant la couche 1 de la pile protocolaire dans le modèle
réseau OSI, qui consiste en général en les pilotes~--- \lang{``drivers''}~---
des émetteurs~/ récepteurs radio, dans le cadre des réseaux sans-fil auquel
se consacre la présente thèse.

\item[PRR] \lang{``Packet Reception Rate''} \\
Acronyme anglo-saxon, se traduisant en français par \nom{<<~Taux de
Réception de Paquets~>>}. Désigne la fraction de paquets transmise avec
succès de leur émetteur à leur destinataire final. Il s'agit bien évidemment
d'un critère essentiel de QdS.

\end{description}

%%%%%%%%%%%%%%%%%%%%%%%%%%%%%%%%%%%%%%%%%%%%%%%%%%%%%%%%%%%%%%%%%%%%%%%%%%%%%

\section*{Q}

\begin{description}

\item[QdS] \nom{<<~Qualité de Service~>>} \\
Acronyme francophone, traduction de l'anglo-saxon \lang{``Quality of
Service''}.

\item[QoS] \lang{``Quality of Service'} \\
Acronyme anglo-saxon, se traduisant en français par \nom{<<~Qualité de
Service~>>}.

\end{description}

%%%%%%%%%%%%%%%%%%%%%%%%%%%%%%%%%%%%%%%%%%%%%%%%%%%%%%%%%%%%%%%%%%%%%%%%%%%%%

\section*{R}

\begin{description}

\item[RDC] \lang{``Radio Duty Cycle''} \\
Acronyme anglo-saxon, se traduisant en français par \nom{<<~Cycle
d'Activité de la Radio~>>}. Désigne la stratégie d'activation~/
désactivation cyclique de l'émetteur~/ récepteur radio d'un n{\oe}ud de
réseau de capteurs sans-fil, afin d'économiser au mieux l'énergie de
ce n{\oe}ud (la radio étant le circuit le plus gourmand en énergie dans
de tels appareils), tout en assurant la bonne transmission des informations
sur le réseau (c'est-à-dire que la radio doit rester active aux moments
nécessaires). Cette stratégie dait l'objets de protocoles, en général
intégrés aux protocoles MAC, lesquels deviennent alors des protocoles
MAC~/RDC~--- assurant ces deux fonctions liées et complémentaires.

Par extension, \lang{``duty cycle''} désigne également la fraction
cyclique d'activation de la radio (c-à-d., pour simplifier, le pourcentage
d'activité radio) résultant de l'exécution d'un protocole RDC donné.

\item[RFD] \lang{``Reduced Function Device''} \\
Acronyme anglo-saxon, issu du standard IEEE 802.15.4, désignant un matériel
aux ressources limitées, incapable de jouer le rôle de coordinateur de
PAN~--- et par conséquent, limité à celui de <<~noeud-feuille~>>, se
contentant en général d'interagir avec l'environnement physique avec
ses capteurs et (éventuellement) ses actionneurs, et de transmettre
les données physiques ainsi reçues sur le réseau, à l'exclusion de
toute tâche (plus) complexe. Ce terme est opposé à celui de FFD.

\item[RIOT OS] \ \\
L'une des plates-formes logicielles parmi les plus avancées dans le domaine
des WSN. Faisant l'objet d'un développement et d'une diffusion rapide, elle
est au centre de nombreux travaux et études dans le présente thèse.

\item[RSSI] \lang{``Received Signal Strength Indicator''} \\
Acronyme anglo-saxon, désignant la puissance efficace du signal radio lors
d'une transmission donnée, notamment par rapport au bruit de fond ambiant.
Il s'agit d'un indicateur important de qualité de service.

\end{description}

%%%%%%%%%%%%%%%%%%%%%%%%%%%%%%%%%%%%%%%%%%%%%%%%%%%%%%%%%%%%%%%%%%%%%%%%%%%%%

\section*{S}

\begin{description}

\item[S-CoSenS] \ \\
Protocole MAC~/ RDC, conçu au sein de notre équipe de recherche (Madynes).
Dérivé évolué de CoSenS, son implantation et son amélioration, au sein
de plates-formes logicielles adaptées (comme RIOT OS) ont constitué le
c{\oe}ur des travaux effectués au sein de la présente thèse. Il semble
notamment offir des propriétés de QdS supérieures à celles de ContikiMAC,
notamment en présence d'un trafic réseau intense.

\item[SoC] \lang{``System-on-Chip''} \\
Acronyme anglo-saxon, se traduisant en français par \nom{<<~Système sur
une puce~>>}. Désigne un système informatique complet regroupé sur un
seul circuit intégré. Par rapport à un microcontrôleur, la logique est
poussée à son maximum, le SoC étant normalement à lui seul un matériel
informatique complet et auto-suffisant, là où le microcontrôleur est plutôt
destiné à être associé à d'autres circuits ou éléments annexes pour former
un autre appareil (système informatique, ou appareil industriel,
électroménager, etc.)

\item[Sensor] \ \\
Terme anglo-saxon, désignant en français un \nom{capteur} capable de mesurer
un phénomène physique de l'environnement. La présence de tels capteurs sur
un n{\oe}ud équipé d'un émetteur~/ récepteur radio est ce qui le fait
désigner sous le terme de <<~capteur sans-fil~>> (\lang{``Wireless
Sensor''} en anglais).

\item[SNR] \lang{``Signal/Noise Ratio''} \\
Acronyme anglo-saxon, se traduisant en français par \nom{<<~Rapport Signal
sur Bruit~>>}. Désigne la force d'un signal utile par rapport au bruit
ambiant (interférences) sur un médium radio donné. Ce paramètre physique
a donc bien évidemment une influence primordiale sur la QdS. La nature
même du médium radio rend ce paramètre SNR extrêmement variable dans
l'espace et le temps, ce qui rend le dit médium peu fiable, \emph{et}
difficile à simuler par logiciel.

\item[SPI] \lang{``Serial Peripheral Interface''} \\
Acronyme anglo-saxon, désignant une technologie de bus série, reliant des
circuits au sein d'un même appareil électronique (parfois même au sein
d'un même microcontrôleur). Souvent employé au sein des n{\oe}uds de
réseaux de capteurs sans-fil pour relier le microcontrôleur central
à l'émetteur~/ récepteur radio.

\end{description}

%%%%%%%%%%%%%%%%%%%%%%%%%%%%%%%%%%%%%%%%%%%%%%%%%%%%%%%%%%%%%%%%%%%%%%%%%%%%%

\section*{T}

\begin{description}

\item[TDMA] \lang{``Time Division Multiple Access''} \\
Acronyme anglo-saxon, désignant une procédure, basée sur le multiplexage
temporel, destinée à gérer l'envoi correct d'un paquet de données sur le
médium radio.

Employée par de nombreux protocoles MAC alternatifs, conçus par la
communauté académique ou industrielle.

\item[Transceiver] \ \\
Terme anglo-saxon, diminutif de \lang{``Radio Transceiver''}, désignant
en français un \nom{émetteur~/ récepteur radio}.

\item[TRP] \nom{<<~Taux de Réception de Paquets~>>} \\
Acronyme francophone, se traduisant en anglais par \lang{``Packet Reception
Rate''}. Désigne la fraction de paquets transmise avec succès de leur
émetteur à leur destinataire final. Il s'agit bien évidemment d'un critère
essentiel de QdS.

\end{description}

%%%%%%%%%%%%%%%%%%%%%%%%%%%%%%%%%%%%%%%%%%%%%%%%%%%%%%%%%%%%%%%%%%%%%%%%%%%%%

\section*{W}

\begin{description}

\item[WAN] \lang{``Wide Area Network''} \\
Acronyme anglo-saxon, se traduisant en français par <<~Réseau de Vaste
\'Etendue~>>. Désigne en général l'Internet.

\item[WSN] \lang{``Wireless Sensor Network''} \\
Acronyme anglo-saxon, se traduisant en français par \nom{<<~Réseau de
Capteurs Sans-Fil~>>}.

\end{description}

%%%%%%%%%%%%%%%%%%%%%%%%%%%%%%%%%%%%%%%%%%%%%%%%%%%%%%%%%%%%%%%%%%%%%%%%%%%%%

%\section*{X}

%\begin{description}

%\item[XXX] \ \\
%Définition de l'élément XXX.

%\item[XXX] \ \\
%Définition de l'élément XXX.

%\end{description}


%%%%%%%%%%%%%%%%%%%%%%%%%%%%%%%%%%%%%%%%%%%%%%%%%%%%%%%%%%%%%%%%%%%%%%%%%%%%%
%%%                           FIN DU GLOSSAIRE                            %%%
%%%%%%%%%%%%%%%%%%%%%%%%%%%%%%%%%%%%%%%%%%%%%%%%%%%%%%%%%%%%%%%%%%%%%%%%%%%%%




